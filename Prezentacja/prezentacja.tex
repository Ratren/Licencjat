\documentclass[11pt]{beamer}
\usepackage[polish]{babel}
\pagestyle{empty}
\usepackage{minted}
\usepackage[T1]{fontenc}

\title{Ocena oraz porównanie wydajności i łatwości użycia biblioteki distributed-ranges na tle innych modeli programowania równoległego}
\author{Rafał Lenart}

\begin{document}

\maketitle
\begin{frame}
	\tableofcontents
\end{frame}

\section{Wstęp}
\subsection{Cel projektu}
\begin{frame}{\subsecname}
	Celem przeprowadzanej oceny jest określenie przypadków użycia oraz ewaluacja użyteczności  i wydajności prezentowanej biblioteki, porównując ją do innych narzędzi służacych do programowania równoległego.

	Opinia ta zostanie wydana na podstawie implementacji wybranych algorytmów i pomiarze 
	wydajności czasowej i pamięciowej programów. 
	Dodatkowo,	zważając na fakt, iż jedną z głównych intencji twórców biblioteki jest 
	usprawnienie procesu pisania kodu dla programistów, krytyce zostanie poddana łatwość 
	pracy z danym narzędziem oraz ilość napisanego kodu.
\end{frame}

\begin{frame}
	\subsection{Plan działania}
	\frametitle{Plan działania}
	
	\begin{enumerate}
		\item Wybór podobnych do siebie technologii/narzędzi użytych do porównania.
		\item Dobór algorytmów do implementacji.
		\item Implementacja, pisanie kodu.
		\item Pomiar prędkości oraz zużycia pamięci. Wyznaczenie innych kryteriów oceny.
		\item Dokonanie porównania technologii z wzięciem pod uwagę wszystkich kryteriów. 
	\end{enumerate}	
		
\end{frame}

\begin{frame}
	\section{Użyte technologie}
	\begin{center}
	\usebeamerfont{title}\insertsectionhead\par%
	\end{center}
\end{frame}

\begin{frame}
	\subsection{aa}
\end{frame}

\end{document}